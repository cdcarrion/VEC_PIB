\documentclass[11pt,]{article}
\usepackage[left=1in,top=1in,right=1in,bottom=1in]{geometry}
\newcommand*{\authorfont}{\fontfamily{phv}\selectfont}
\usepackage[]{mathpazo}


  \usepackage[T1]{fontenc}
  \usepackage[utf8]{inputenc}



\usepackage{abstract}
\renewcommand{\abstractname}{}    % clear the title
\renewcommand{\absnamepos}{empty} % originally center

\renewenvironment{abstract}
 {{%
    \setlength{\leftmargin}{0mm}
    \setlength{\rightmargin}{\leftmargin}%
  }%
  \relax}
 {\endlist}

\makeatletter
\def\@maketitle{%
  \newpage
%  \null
%  \vskip 2em%
%  \begin{center}%
  \let \footnote \thanks
    {\fontsize{18}{20}\selectfont\raggedright  \setlength{\parindent}{0pt} \@title \par}%
}
%\fi
\makeatother


\renewcommand*\thetable{A.\arabic{table}}
\renewcommand*\thefigure{A.\arabic{figure}}


\setcounter{secnumdepth}{0}


\usepackage{graphicx,grffile}
\makeatletter
\def\maxwidth{\ifdim\Gin@nat@width>\linewidth\linewidth\else\Gin@nat@width\fi}
\def\maxheight{\ifdim\Gin@nat@height>\textheight\textheight\else\Gin@nat@height\fi}
\makeatother
% Scale images if necessary, so that they will not overflow the page
% margins by default, and it is still possible to overwrite the defaults
% using explicit options in \includegraphics[width, height, ...]{}
\setkeys{Gin}{width=\maxwidth,height=\maxheight,keepaspectratio}

\title{La relación a largo plazo y la causalidad entre las exportacioens
mineras, la producción industrial y el crecimiento económico en Perú: Un estudio de caso utilizando un modelo VEC \thanks{La replicacion de los archivos están disponibles en la página Github del
autor (\href{https://github.com/cristian1512}{github.com/cristian1512})}  }



\author{\Large Cristian Carrión\vspace{0.05in} \newline\normalsize\emph{Escuela Politécnica Nacional}   \and \Large Christian Socasi\vspace{0.05in} \newline\normalsize\emph{Escuela Politécnica Nacional}  }


\date{}

\usepackage{titlesec}

\titleformat*{\section}{\normalsize\bfseries}
\titleformat*{\subsection}{\normalsize\itshape}
\titleformat*{\subsubsection}{\normalsize\itshape}
\titleformat*{\paragraph}{\normalsize\itshape}
\titleformat*{\subparagraph}{\normalsize\itshape}





\newtheorem{hypothesis}{Hypothesis}
\usepackage{setspace}

\makeatletter
\@ifpackageloaded{hyperref}{}{%
\ifxetex
  \PassOptionsToPackage{hyphens}{url}\usepackage[setpagesize=false, % page size defined by xetex
              unicode=false, % unicode breaks when used with xetex
              xetex]{hyperref}
\else
  \PassOptionsToPackage{hyphens}{url}\usepackage[unicode=true]{hyperref}
\fi
}

\@ifpackageloaded{color}{
    \PassOptionsToPackage{usenames,dvipsnames}{color}
}{%
    \usepackage[usenames,dvipsnames]{color}
}
\makeatother
\hypersetup{breaklinks=true,
            bookmarks=true,
            pdfauthor={Cristian Carrión (Escuela Politécnica Nacional) and Christian Socasi (Escuela Politécnica Nacional)},
             pdfkeywords = {},  
            pdftitle={La relación a largo plazo y la causalidad entre las exportacioens
mineras, la producción industrial y el crecimiento económico en Perú: Un estudio de caso utilizando un modelo VEC},
            colorlinks=true,
            citecolor=blue,
            urlcolor=blue,
            linkcolor=magenta,
            pdfborder={0 0 0}}
\urlstyle{same}  % don't use monospace font for urls

% set default figure placement to htbp
\makeatletter
\def\fps@figure{htbp}
\makeatother

\usepackage{hyperref}
\usepackage[spanish]{babel}
\usepackage{caption}
\usepackage{multirow}


% add tightlist ----------
\providecommand{\tightlist}{%
\setlength{\itemsep}{0pt}\setlength{\parskip}{0pt}}

\begin{document}
	
% \pagenumbering{arabic}% resets `page` counter to 1 
%%\renewcommand*{\thepage}{A--\arabic{page}}
%
% \maketitle

{% \usefont{T1}{pnc}{m}{n}
\setlength{\parindent}{0pt}
\thispagestyle{plain}
{\fontsize{18}{20}\selectfont\raggedright 
\maketitle  % title \par  

}

{
   \vskip 13.5pt\relax \normalsize\fontsize{11}{12} 
\textbf{\authorfont Cristian Carrión} \hskip 15pt \emph{\small Escuela Politécnica Nacional}   \par \textbf{\authorfont Christian Socasi} \hskip 15pt \emph{\small Escuela Politécnica Nacional}   

}

}








\begin{abstract}

    \hbox{\vrule height .2pt width 39.14pc}

    \vskip 8.5pt % \small 

\noindent El medio ambiente se convirtió en una cuestión de importancia
internacional en 1972, cuando se celebró en Estocolmo la Conferencia de
las Naciones Unidas sobre el Medio Ambiental. La relación entre
crecimiento económico y medio ambiente ha sido polémica durante mucho
tiempo, por los


    \hbox{\vrule height .2pt width 39.14pc}


\end{abstract}


\vskip 6.5pt


\noindent  \hypertarget{introduccion}{%
\section{Introducción}\label{introduccion}}

Durante las últimas décadas, el debate de la relación entre crecimiento
económico y el medio ambiente se ha intensificado a nivel teórico y
mediante aplicaciones empíricas en diversas partes del mundo. La
actividad productiva, extractiva y consumista de la sociedad durante
siglos dejó de considerar que vivir en un ambiente sano es una necesidad
indispensable para los individuos. Desde 1990, el vínculo entre las
emisiones de CO 2 y El crecimiento económico ha sido ampliamente
estudiado. Prueba de ello es la cuantiosa proliferación de informes
ambientales y ecológicos, cumbres del planeta, o posiciones ideológicas
adoptadas.

\hypertarget{revision-de-la-literatura}{%
\section{Revisión de la literatura}\label{revision-de-la-literatura}}

En la literatura, existen tres categorías de estudio sobre la relación
entre emisiones de CO2 y crecimiento económico: La primera categoría se
centra en la relación. entre contaminantes ambientales y crecimiento
económico e investiga la validez de los Kuznets ambientales (1955)
curva. La segunda categoría investiga el relación entre crecimiento
económico y consumo de energía, y finalmente la tercera categoría
examina la relación dinámica entre crecimiento económico, energía
consumo y contaminantes ambientales(Vásquez Sánchez \& García Rendón,
2003).

\hypertarget{datos-y-metodologia}{%
\section{Datos y Metodología}\label{datos-y-metodologia}}

\hypertarget{datos}{%
\subsection{Datos}\label{datos}}

En este estudio utilizamos datos anuales del Exportaciones Mineras
Metálicas \emph{(XMIN)}, Producto Interno Bruto Manufacturero
\emph{(PBIMAN)} y el Producto Interno Bruto \emph{(PBI)} de Italia
durante el periodo de 1994-2016. Todos los datos recopilados del Banco
Central de Perú.

\hypertarget{metodologia}{%
\subsection{Metodología}\label{metodologia}}

La relación entre Exportaciones Mineras Metálicas, Producto Interno
Bruto Manufacturero y el Producto Interno Bruto se puede expresar de la
siguiente manera:

\begin{eqnarray}\label{eq1}
PBI_t=\propto_0+\alpha_1PBIMAN_t+\alpha_2XMIN_t+\varepsilon_T     
\end{eqnarray}

Donde:

\begin{itemize}
\tightlist
\item
  \({PBI}_t =\) Producto Interno Bruto
\item
  \({PBIMAN}_t =\) Producto Interno Bruto Manufacturero
\item
  \({XMIN}_t =\) Exportaciones Mineras Metálicas
\item
  \(\varepsilon_t =\) Ruido Blanco
\end{itemize}

Donde

\begin{itemize}
\tightlist
\item
  \(t\): es la variable de tendencia.
\end{itemize}

Entonces, comenzamos el proceso de estimación a partir del modelo
\ref{eq1}. El propósito de este trabajo es examinar la relación a largo
plazo entre Exportaciones Mineras Metálicas, Producto Interno Bruto
Manufacturero y el Producto Interno Bruto en Perú. El enfoque
metodológico del estudio incluye los siguientes pasos:

\begin{itemize}
\tightlist
\item
  Comprobamos las propiedades de estacionariedad de la serie aplicando
  la prueba de Dickey-Fuller (ADF) aumentada (Dickey and Fuller 1981)
  así como la prueba de Phillips-Perron (PP)(Phillips and Perron 1988).
\item
  Si todas las variables están integradas de orden uno, entonces la
  prueba de cointegración de Johansen (1995) es la más apropiada para
  ser usada.
\item
  El tercer paso consiste en comprobar la relación causal entre las
  variables utilizando las pruebas apropiadas.
\end{itemize}

En la Figura \ref{fig:plot1} se presenta la evolución de las tres
variables, Exportaciones Mineras Metálicas, Producto Interno Bruto
Manufacturero y el Producto Interno Bruto de Perú durante los años
1994-2016. Las estadísticas descriptivas de las variables se muestran en
la Cuadro \ref{tab:descript}.

\begin{figure}
\centering
\includegraphics{figs/graph1.pdf}
\caption{\label{fig:plot1} Representación gráfica de los datos de Perú
(Niveles)}
\end{figure}

\begin{table}[!htbp] \centering 
  \caption{Estadística Descriptiva para las Variables Usadas en el Análisis} 
  \label{tab:descript} 
\small 
\begin{tabular}{@{\extracolsep{5pt}}lccccc} 
\\[-1.8ex]\hline \\[-1.8ex] 
Statistic & \multicolumn{1}{c}{N} & \multicolumn{1}{c}{Mean} & \multicolumn{1}{c}{St. Dev.} & \multicolumn{1}{c}{Min} & \multicolumn{1}{c}{Max} \\ 
\hline 
\hline \\[-1.8ex] 
PBI & 22 & 14,558.98 & 9,255.18 & $-$838.30 & 29,795.98 \\ 
PBIMAN & 22 & 1,526.96 & 2,504.48 & $-$3,851.98 & 5,752.98 \\ 
XMIN & 22 & 900.09 & 2,459.98 & $-$3,678.83 & 5,620.76 \\ 
\hline 
\hline \\[-1.8ex] 
\multicolumn{6}{l}{\footnotesize{\textit{Nota:} Las estadísticas descriptivas corresponden}} \\ 
\multicolumn{6}{l}{\footnotesize{~~~~~~~~~~ al periodo 1990-2014}} \\ 
\multicolumn{6}{l}{\footnotesize{\textit{Elaboración: Los autores}}} \\ 
\end{tabular} 
\end{table}

\hypertarget{la-relacion-causal-entre-el-pib-exportaciones-mineras-metalicas-y-el-producto-interno-bruto-manufacturero-resultados-empiricos}{%
\section{La relación causal entre el PIB, Exportaciones Mineras
Metálicas y el Producto Interno Bruto Manufacturero: Resultados
Empíricos}\label{la-relacion-causal-entre-el-pib-exportaciones-mineras-metalicas-y-el-producto-interno-bruto-manufacturero-resultados-empiricos}}

\hypertarget{pruebas-de-raices-unitarias}{%
\subsection{Pruebas de Raíces
Unitarias}\label{pruebas-de-raices-unitarias}}

Aplicando las pruebas de raíz unitaria del ADF, de Dickey y Fuller
(Dickey and Fuller 1981), y del PP, de Phillips y Perron (Phillips and
Perron 1988), presentamos los resultados en el Cuadro \ref{tab:tb2}. Los
resultados del Cuadro \ref{tab:tb2} mostraron que todas las variables
son estacionarias en su primera diferencias (para la prueba de ADF), es
decir, están integradas de orden uno (es decir, \emph{I (\ref{eq1})}).
Por lo tanto, se continua aplicando el enfoque de cointegración de
Johansen para examinar la relación a largo plazo entre las variables.

\begin{table}[!htbp] \centering 
  \caption{Los resultados de la prueba de raíz de la unidad ADF sobre el crecimiento económico y el CO2 durante 1990-2014} 
  \label{tab:tb2} 
\begin{tabular}{@{\extracolsep{5pt}} lcclccl} 
\\[-1.8ex]\hline 
\hline \\[-1.8ex] 
\\[-1.8ex] &  \multicolumn{3}{c}{\textit{Dickey–Fuller}} &  \multicolumn{3}{c}{\textit{Phillips–Perron}}\\
\\[-1.8ex] Variables & t Valor & p & Estabilidad & t Valor & p & Estabilidad\\
\hline \\[-1.8ex] 
PBI & -1.168 & 0.889 & No Estacionario & -1.829 & 0.967 & No Estacionario \\ 
PBIMAN & -1.627 & 0.714 & No Estacionario & -6.334 & 0.721 & No Estacionario \\ 
XMIN & -1.498 & 0.764 & No Estacionario & -7.139 & 0.667 & No Estacionario \\ 
 &  &  &  &  &  &  \\ 
DPBI & -4.157 & 0.018 & Estacionario & -20.681 & 0.02 & Estacionario \\ 
DPBIMAN & -3.96 & 0.025 & Estacionario & -20.272 & 0.023 & Estacionario \\ 
DXMIN & -3.853 & 0.032 & Estacionario & -12.212 & 0.327 & No Estacionario \\ 
\hline \\[-1.8ex] 
\multicolumn{7}{l}{\footnotesize{D: Primeraa Diferencia}} \\ 
\multicolumn{7}{l}{\footnotesize{\textit{Elaboración: Los autores}}} \\ 
\end{tabular} 
\end{table}

\hypertarget{prueba-de-cointegracion}{%
\subsection{Prueba de Cointegración}\label{prueba-de-cointegracion}}

Dado que el enfoque de Johansen (1988) es sensible a la longitud del
desfase, antes de aplicar la prueba de cointegración, tenemos que
encontrar el orden del modelo VAR. La longitud de desfase óptima se
selecciona por el valor mínimo de los criterios AIC, SC, HQ y FPE. En la
Cuadro \ref{tab:tb3} se presentan los resultados de estos criterios.
Todos los criterios indican que la longitud óptima del rezago es igual a
3. Por lo tanto, el orden del modelo VEC es igual a 2. La prueba de
cointegración de Johansen se basa en la estadística de trazas y en la
estadística de valores propios máximos. El número máximo de vectores de
cointegración es el número de variables menos uno (n-1=3-1=2). Para la
prueba del rastro, la hipótesis nula es que hay vectores de
cointegración r frente a la alternativa de más de r. La hipótesis nula
de la prueba del valor propio máximo sigue siendo la misma que antes;
sin embargo, la alternativa es que haya exactamente vectores de
cointegración r+1. Los resultados de la prueba de cointegración de
Johansen se presentan en el Cuadro \ref{tab:tb4}.

\begin{table}[!htbp] \centering 
  \caption{Criterios de selección del orden de rezagos del VAR (Max=3)} 
  \label{tab:tb3} 
\small 
\begin{tabular}{@{\extracolsep{5pt}} lccc} 
\\[-1.8ex]\hline 
\hline \\[-1.8ex] 
Parámetros & Lag 1 & Lag 2 & Lag 3 \\ 
\hline \\[-1.8ex] 
AIC(n) & 47.206 & 46.87 & 45.67$^{*}$ \\ 
HQ(n) & 47.322 & 47.074 & 45.961$^{*}$ \\ 
SC(n) & 47.803 & 47.915 & 47.163$^{*}$ \\ 
FPE(n) & 3.22e+20 & 2.49e+20 & 9.18e+19$^{*}$ \\ 
\hline \\[-1.8ex] 
\multicolumn{4}{l}{\footnotesize{$^{*}$Indica el orden de retraso seleccionado por el criterio}} \\ 
\multicolumn{4}{l}{\footnotesize{\textit{Elaboración: Los autores}}} \\ 
\end{tabular} 
\end{table}

\begin{table}[!htbp] \centering 
  \caption{Los resultados de la prueba de cointegración del crecimiento económico, las exportaciones mineras y la producción industrial durante 1994-2016} 
  \label{tab:tb4} 
\begin{tabular}{@{\extracolsep{5pt}} clccc} 
\\[-1.8ex]\hline 
\hline \\[-1.8ex] 
 & Hipótesis & Trace Estadístico & p valor & Ecuac. de 
Cointegración \\ 
\hline \\[-1.8ex] 
1 & H0:r=0, H1:r\textgreater 0 & $53.701$ & \textless  0.001 & $0$ \\ 
2 & H0:r=1, H1:r \textgreater 1 & $23.972$ & 0.001637 & $0$ \\ 
3 & H0:r=2, H1:r \textgreater 2 & $4.178$ & 0.050960 & $2$ \\ 
\hline \\[-1.8ex] 
\multicolumn{5}{l}{\footnotesize{Las estadísticas indican 2 ecuaciones de cointegración al nivel del .01 de p-valor}} \\ 
\multicolumn{5}{l}{\footnotesize{\textit{Elaboración: Los autores}}} \\ 
\end{tabular} 
\end{table}

Los resultados del Cuadro \ref{tab:tb4} (estadísticas de pruebas de
trazas) apoyan la presencia de un vector cointegrador con un nivel de
significación del .01. Concluimos que hay una fuerte evidencia de
cointegración a largo plazo entre las variables examinadas. Las
ecuaciones de cointegración a largo plazo se muestran a continuación:

\begin{eqnarray}\label{eq2}
PBI_{t-1}=8.485XMIN_{t-1}+192813.94 \\
\label{eq3}
PBIMAN_{t-1}=1.237XMIN_{t-1}+29796.04
\end{eqnarray}

Basándose en la ecuación \ref{eq2} de cointegración anterior, el estudio
concluye que, los impactos a largo plazo de las Exportaciones Mineras en
el PIB son positivos y significativos, es decir, se mueven juntos en la
misma dirección. Mientras que para al ecuación \ref{eq3} de
cointegración se concluye que los impactos a largo plazo de las
Exportaciones Mineras en el PIB Manufacturero son positivos y
significativos, es decir, se mueven juntos en la misma dirección.

\hypertarget{estimacion-y-analisis-del-vecm}{%
\subsection{Estimación y análisis del
VECM}\label{estimacion-y-analisis-del-vecm}}

\begin{table}[!htbp] \centering 
  \caption{Los resultados de la estimación del modelo VEC} 
  \label{tab:tb6} 
\small 
\begin{tabular}{@{\extracolsep{5pt}} lccc} 
\\[-1.8ex]\hline 
\hline \\[-1.8ex] 
\\[-1.8ex] & $\Delta PBI$  &  $\Delta PBIMAN$ & $\Delta XMIN$\\
\hline \\[-1.8ex] 
ETC1 & -0.298 & -0.153$^{*}$ & -0.247$^{***}$ \\ 
ETC2 & 4.659 & 1.615$^{*}$ & 3.002$^{***}$ \\ 
Intercepto & -77623.26 & -17472.577 & -41863.975 \\ 
 $\Delta$PBI(-1) & 0.139 & 0.053 & -0.081 \\ 
$\Delta$PBIMAN(-1) & -5.308$^{**}$ & -1.937$^{***}$ & -2.103$^{***}$ \\ 
$\Delta$XMIN(-1) & 4.2$^{***}$ & 1.153$^{***}$ & 1.699$^{***}$ \\ 
$\Delta$PBI(-2) & 0.562 & 0.023 & 0.029 \\ 
$\Delta$PBIMAN(-2) & -4.067$^{**}$ & -1.188$^{*}$ & -1.513$^{***}$ \\ 
$\Delta$XMIN(-2) & 4.042$^{**}$ & 1.032$^{*}$ & 0.841$^{***}$ \\ 
\hline \\[-1.8ex] 
\multicolumn{4}{l}{\footnotesize{\textit{Nota:} $^{***}$, $^{**}$ y $^{*}$ indica el nivel significativo al .01, .05 y .1 respectivamente}} \\ 
\multicolumn{4}{l}{\footnotesize{\textit{Elaboración: Los autores}}} \\ 
\end{tabular} 
\end{table}

\begin{table}[!htbp] \centering 
  \caption{Resultados de las pruebas de diagnóstico del VEC(2)} 
  \label{tab:tb5} 
\small 
\begin{tabular}{@{\extracolsep{5pt}} clcc} 
\\[-1.8ex]\hline 
\hline \\[-1.8ex] 
 & Test de Diagnóstico & Estadística & p-valor \\ 
\hline \\[-1.8ex] 
1 & Test J-B & 7.8761 & 0.2473 \\ 
2 & LM Autocorrelación & 84.698 & 0.9939 \\ 
3 & Heterocedasticidad & 90 & 1 \\ 
4 & Estabilidad & Raíces fuera del círculo unitario & - \\ 
\hline \\[-1.8ex] 
\multicolumn{4}{l}{\footnotesize{\textit{Nota:} La Ho nula de J-B es la normalidad residual, la Ho }} \\ 
\multicolumn{4}{l}{ \footnotesize{es no autocorrelación, la Ho de de la prueba ARCH es heterocedasticidad,}} \\ 
\multicolumn{4}{l}{ \footnotesize{la estabilidad VEC revela que 4 raíces tienen unmódulo menor que uno}} \\ 
\multicolumn{4}{l}{\footnotesize{y se encuentran dentro del círculo de la unidad.}} \\ 
\multicolumn{4}{l}{\footnotesize{\textit{Elaboración: Los autores}}} \\ 
\end{tabular} 
\end{table}

ADASASDADADASDDA

\begin{table}[!htbp] \centering 
  \caption{Los resultados de la causalidad de Granger} 
  \label{tab:tb7} 
\small 
\begin{tabular}{@{\extracolsep{5pt}} lcccl} 
\\[-1.8ex]\hline 
\hline \\[-1.8ex] 
\\[-1.8ex] VarDep &   &  &  &Causalidad\\
\\[-1.8ex] & $\Delta PBI$  &  $\Delta PBIMAN$ & $\Delta XMIN$ &\\
\hline \\[-1.8ex] 
$\Delta$PBI & - & 6.98945$^{**}$ & 13.51599$^{***}$ & $\Delta$PBI $\leftarrow$ $\Delta$PBIMAN, $\Delta$PBI $\leftarrow$ DXMIN \\ 
 &  & (0.0304) & (0.0012) &  \\ 
$\Delta$PBIMAN & 0.321026 & - & 10.062$^{***}$ & $\Delta$PBIMAN $\leftarrow$ $\Delta$XMIN \\ 
 & (0.8517) &  & (0.0065) &  \\ 
$\Delta$XMIN & 1.297252 & 39.44536$^{***}$ & - & $\Delta$XMIN $\leftarrow$ $\Delta$PBIMAN \\ 
 & (0.5228) & (0.0000) &  &  \\ 
\hline \\[-1.8ex] 
\multicolumn{5}{l}{\footnotesize{\textit{Nota: $^{***}$ y $^{**}$} indica el nivel significativo al .01 y .05, respectivamente; el nivel de probabilidad están}} \\ 
\multicolumn{5}{l}{\footnotesize{entre paréntesis; $\leftarrow$ denota una causalidad unidireccional; $\Delta$ denota primera diferencia.}} \\ 
\multicolumn{5}{l}{\footnotesize{\textit{Elaboración: Los autores}}} \\ 
\end{tabular} 
\end{table}

Con el fin de analizar la dinámica efectos del modelo que responden a
ciertos choques, así como cómo son los efectos entre las tres variables,
un análisis más profundo se realiza a través de la función de respuesta
a los impulsos y la varianza descomposición basada en el VECM, y los
resultados de 10 períodos se obtienen.

\newpage

\hypertarget{referencias}{%
\section{Referencias}\label{referencias}}

\setlength{\parindent}{-0.2in}
\setlength{\leftskip}{0.2in}
\setlength{\parskip}{8pt}
\vspace*{-0.2in}

\noindent

\hypertarget{refs}{}
\leavevmode\hypertarget{ref-Dickey1981}{}%
Dickey, David A., and Wayne A. Fuller. 1981. ``Likelihood Ratio
Statistics for Autoregressive Time Series with a Unit Root.''
\emph{Econometrica} 49 (4): 1057. \url{https://doi.org/10.2307/1912517}.

\leavevmode\hypertarget{ref-Phillips1988}{}%
Phillips, Peter C. B., and Pierre Perron. 1988. ``Testing for a Unit
Root in Time Series Regression.'' \emph{Biometrika} 75 (2): 335.
\url{https://doi.org/10.2307/2336182}.




\newpage
\singlespacing 
\end{document}
