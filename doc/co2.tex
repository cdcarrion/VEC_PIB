\documentclass[11pt,]{article}
\usepackage[left=1in,top=1in,right=1in,bottom=1in]{geometry}
\newcommand*{\authorfont}{\fontfamily{phv}\selectfont}
\usepackage[]{mathpazo}


  \usepackage[T1]{fontenc}
  \usepackage[utf8]{inputenc}



\usepackage{abstract}
\renewcommand{\abstractname}{}    % clear the title
\renewcommand{\absnamepos}{empty} % originally center

\renewenvironment{abstract}
 {{%
    \setlength{\leftmargin}{0mm}
    \setlength{\rightmargin}{\leftmargin}%
  }%
  \relax}
 {\endlist}

\makeatletter
\def\@maketitle{%
  \newpage
%  \null
%  \vskip 2em%
%  \begin{center}%
  \let \footnote \thanks
    {\fontsize{18}{20}\selectfont\raggedright  \setlength{\parindent}{0pt} \@title \par}%
}
%\fi
\makeatother


\renewcommand*\thetable{A.\arabic{table}}
\renewcommand*\thefigure{A.\arabic{figure}}


\setcounter{secnumdepth}{0}


\usepackage{graphicx,grffile}
\makeatletter
\def\maxwidth{\ifdim\Gin@nat@width>\linewidth\linewidth\else\Gin@nat@width\fi}
\def\maxheight{\ifdim\Gin@nat@height>\textheight\textheight\else\Gin@nat@height\fi}
\makeatother
% Scale images if necessary, so that they will not overflow the page
% margins by default, and it is still possible to overwrite the defaults
% using explicit options in \includegraphics[width, height, ...]{}
\setkeys{Gin}{width=\maxwidth,height=\maxheight,keepaspectratio}

\title{La relación a largo plazo y la causalidad entre las exportaciones
mineras, la producción industrial y el crecimiento económico en Perú: Un estudio de caso utilizando un modelo VEC \thanks{La replicación de los archivos están disponibles en la página Github del
autor (\href{https://github.com/cristian1512}{github.com/cristian1512})}  }



\author{\Large Cristian Carrión\vspace{0.05in} \newline\normalsize\emph{Escuela Politécnica Nacional}   \and \Large Christian Socasi\vspace{0.05in} \newline\normalsize\emph{Escuela Politécnica Nacional}  }


\date{}

\usepackage{titlesec}

\titleformat*{\section}{\normalsize\bfseries}
\titleformat*{\subsection}{\normalsize\itshape}
\titleformat*{\subsubsection}{\normalsize\itshape}
\titleformat*{\paragraph}{\normalsize\itshape}
\titleformat*{\subparagraph}{\normalsize\itshape}





\newtheorem{hypothesis}{Hypothesis}
\usepackage{setspace}

\makeatletter
\@ifpackageloaded{hyperref}{}{%
\ifxetex
  \PassOptionsToPackage{hyphens}{url}\usepackage[setpagesize=false, % page size defined by xetex
              unicode=false, % unicode breaks when used with xetex
              xetex]{hyperref}
\else
  \PassOptionsToPackage{hyphens}{url}\usepackage[unicode=true]{hyperref}
\fi
}

\@ifpackageloaded{color}{
    \PassOptionsToPackage{usenames,dvipsnames}{color}
}{%
    \usepackage[usenames,dvipsnames]{color}
}
\makeatother
\hypersetup{breaklinks=true,
            bookmarks=true,
            pdfauthor={Cristian Carrión (Escuela Politécnica Nacional) and Christian Socasi (Escuela Politécnica Nacional)},
             pdfkeywords = {},  
            pdftitle={La relación a largo plazo y la causalidad entre las exportaciones
mineras, la producción industrial y el crecimiento económico en Perú: Un estudio de caso utilizando un modelo VEC},
            colorlinks=true,
            citecolor=blue,
            urlcolor=blue,
            linkcolor=magenta,
            pdfborder={0 0 0}}
\urlstyle{same}  % don't use monospace font for urls

% set default figure placement to htbp
\makeatletter
\def\fps@figure{htbp}
\makeatother

\usepackage{hyperref}
\usepackage[spanish]{babel}
\usepackage{caption}
\usepackage{multirow}
\usepackage{amsmath}


% add tightlist ----------
\providecommand{\tightlist}{%
\setlength{\itemsep}{0pt}\setlength{\parskip}{0pt}}

\begin{document}
	
% \pagenumbering{arabic}% resets `page` counter to 1 
%%\renewcommand*{\thepage}{A--\arabic{page}}
%
% \maketitle

{% \usefont{T1}{pnc}{m}{n}
\setlength{\parindent}{0pt}
\thispagestyle{plain}
{\fontsize{18}{20}\selectfont\raggedright 
\maketitle  % title \par  

}

{
   \vskip 13.5pt\relax \normalsize\fontsize{11}{12} 
\textbf{\authorfont Cristian Carrión} \hskip 15pt \emph{\small Escuela Politécnica Nacional}   \par \textbf{\authorfont Christian Socasi} \hskip 15pt \emph{\small Escuela Politécnica Nacional}   

}

}








\begin{abstract}

    \hbox{\vrule height .2pt width 39.14pc}

    \vskip 8.5pt % \small 

\noindent Este documento investiga la relación entre la exportación minera, la
producción industrial y el crecimiento económico en la Republica del
Perú utilizando los datos correspondientes a las variables del Producto
Bruto Interno (PBI), Valores FOB de las exportaciones mineras metálicas
(XMIN) y el Producto Bruto Interno manufacturero (PBIMAN)
correspondientes a los años de 1994-2016, se escoge este periodo debido
a que a partir de 1993 rige la nueva Constitución Política del Perú, que
permitió el mayor ingreso de inversiones a los diferentes sectores de la
economía nacional. La técnica de cointegración multivariante se ha
empleado para ver la relación de equilibrio a largo plazo entre las
variables. Además, se ha adoptado la causalidad de Granger basada en el
modelo de corrección de errores de vectores (VECM) para ver la
causalidad a corto y largo plazo entre las variables. Los resultados de
la cointegración confirman que las exportaciones de minerales, la
producción industrial y el crecimiento económico están cointegrados, lo
que indica una existencia de una relación de equilibrio a largo plazo
entre variables Del mismo modo, el resultado de causalidad de Granger de
VECM sostiene que existe una relación de causalidad de Granger a largo
plazo que va desde el crecimiento económico y la producción industrial
hasta la exportación de minerales del Perú.


    \hbox{\vrule height .2pt width 39.14pc}


\end{abstract}


\vskip 6.5pt

{
\hypersetup{linkcolor=black}
\setcounter{tocdepth}{2}
\tableofcontents
}

\noindent  \newpage

\hypertarget{introduccion}{%
\section{Introducción}\label{introduccion}}

El Perú es el país con mayor atractivo para la inversión minera en la
región. Debido a la concentración de inversiones en el sector minero y a
los altos precios internacionales, los minerales metálicos han llegado a
representar el 55\% de las exportaciones totales del país al 2016,
teniendo un impacto directo sobre el producto bruto interno (Instituto
Peruano de Economía, 2017).

La relación bilateral entre varias variables hace más complejo el
proceso de un modelo econométrico. Uno de los pasos principales para
modelar una determinada relación entre variables es la especificación
del modelo econométrico. La elección del método depende del propósito de
la evaluación y la evaluación de la disponibilidad de datos. Modelos
como VAR y VECM se usan con mayor frecuencia en el tratamiento de tales
relaciones, que tienen más de una variable endógena. Un factor
preocupante es el impacto ambiental de la minería, puesto que ya en los
años 80 era señalada como la actividad económica más contaminante. Por
ello, se han establecido programas de evaluación de pasivos ambientales
y planes de adecuación ambiental para la gran minería. Se estima en
alrededor de US\$ 977,1 millones la inversión necesaria para mitigar la
contaminación ambiental producida por las unidades mineras operativas.
En cuanto a la mediana y pequeña minería, se estima que existe un saldo
de pasivos ambientales de aproximadamente US\$ 181,4 millones,
principalmente por contaminación en cuencas petrolíferas y lavaderos de
oro. No obstante, la indefinición de derechos de propiedad reduce los
incentivos para que las normas ambientales sean aplicadas y dificulta la
fiscalización.

El presente trabajo tiene por objetivo principal: determinar cuál ha
tenido mayor incidencia sobre el crecimiento económico en el Perú en el
periodo 1994-2016. Los objetivos específicos son: 1) Determinar la
existencia de una relación de equilibrio en el largo plazo entre las
exportaciones mineras metálicas, la producción industrial y el
crecimiento económico, para hallar una sincronización en el tiempo que
refleje una relación confiable y determinante; 2) Determinar la
direccionalidad de la relación causal entre las exportaciones mineras
metálicas, la producción industrial y el crecimiento económico.

\hypertarget{revision-empirica}{%
\section{Revisión empírica}\label{revision-empirica}}

En base a la investigación realizada por Auro Kumar, Dukhabandhu Sahoo y
Naresh Chandra (Sahoo, Sahoo, and Sahu 2014) en la cual se investigaron
la relación entre la exportación minera, la producción industrial y el
crecimiento económico en la India utilizando datos de series temporales
anuales de 1981 a 2010, en la cual se la realizo mediante la metodología
de VECM y los resultados de la cointegración multivariante sugieren que
existe una relación de equilibrio a largo plazo entre la exportación
minera, la producción industrial y el crecimiento económico.

Boriss Siliverstovs y Dierk Herzer (Siliverstovs and Herzer 2007) en su
estudio de Exportaciones manufactureras, exportaciones mineras y
crecimiento: análisis de cointegración y causalidad para Chile
(1960--2001) utiliza la técnica de cointegración de Johansen para
examinar los efectos de la productividad de las exportaciones de
manufacturas y minería en el contexto de la hipótesis de crecimiento
basada en la exportación utilizando los datos de series temporales para
1960--2001 mostraron que existe una relación a largo plazo entre el
capital, el trabajo, las importaciones de bienes de capital, las
exportaciones de manufacturas, las exportaciones mineras.

Kegomoditswe Koitsiw y Tsuyoshi Adachi (Koitsiwe and Adachi 2015)
analizaron en su investigación la relación entre ingresos mineros,
consumo del gobierno, tipo de cambio y crecimiento económico en Botswana
con datos trimestrales de 1994 a 2012 mediante el uso de modelado
autorregresivo (VAR) cuyos resultados indican que los ingresos mineros
tienen un impacto significativo en el crecimiento económico y el consumo
del gobierno.

David I. Stern (Stern 2018) mostros en su estudio sobre la contribución
del sector minero a la sostenibilidad en los países en desarrollo
utilizando un modelo de autorregresión vectorial (VAR) para cada uno de
los 19 países en desarrollo no pertenecientes a la OPEP con grandes
sectores mineros, cuyos resultados fueron que no es posible rechazar la
hipótesis de que el sector minero resta valor a la sostenibilidad en los
países en desarrollo.

El resumen de las investigaciones realizadas se encuentra en el Cuadro
\ref{tab:lit}.

\begin{table}[h]\centering 
  \caption{Resumen de los resultados de las pruebas de causalidad de estudios anteriores} 
  \label{tab:lit} 
\tiny
\begin{tabular}{lllll}
\hline
Investigación                                                                                                                                                               & Autor/es                                                                                     & Periodo                                                                                                                        & Metodología & Resultados                                                                                                                                                                                                            \\ \hline
\begin{tabular}[c]{@{}l@{}}Mining export, industrial production \\ and economic growth:  A cointegration \\ and causality analysis for India (2014)\end{tabular}             & \begin{tabular}[c]{@{}l@{}}Auro Kumar, \\ Dukhabandhu Sahoo \\ y Naresh Chandra\end{tabular} & \begin{tabular}[c]{@{}l@{}}1981-2010\\ (anuales)\end{tabular}                                                               & VEC         & \begin{tabular}[c]{@{}l@{}}Existe una relación de equilibrio a largo\\  plazo entre la exportación minera, la\\ producción industrial y el crecimiento\\ económico\end{tabular}                                       \\ 
\begin{tabular}[c]{@{}l@{}}Exportaciones  manufactureras, \\ exportaciones mineras y crecimiento: \\ análisis decointegración y causalidad \\ para Chile (2007)\end{tabular} & \begin{tabular}[c]{@{}l@{}}Boriss Siliverstovs \\ y Dierk Herzer\end{tabular}                & \begin{tabular}[c]{@{}l@{}}1960–2001\\ (anuales)\end{tabular}                                                               & VEC         & \begin{tabular}[c]{@{}l@{}}Existe una relación a largo plazo entre \\ el capital, el trabajo, las importaciones \\ de bienes de capital, las exportaciones de\\ manufacturas, las exportaciones mineras.\end{tabular} \\
\begin{tabular}[c]{@{}l@{}}La  relación entre ingresos mineros,\\ consumo del gobierno, tipo de cambio y\\ crecimiento económico en Botswana \\ (2015)\end{tabular}          & \begin{tabular}[c]{@{}l@{}}Kegomoditswe\\ Koitsiw y Tsuyoshi\\ Adachi\end{tabular}           & \begin{tabular}[c]{@{}l@{}}1994-2012\\ (trimestrales)\end{tabular}                                                          & VAR         & \begin{tabular}[c]{@{}l@{}}Los ingresos mineros tienen un  impacto\\  significativo en el  crecimiento \\ económico y el  consumo del gobierno.\end{tabular}                                                          \\
\begin{tabular}[c]{@{}l@{}}La contribución del sector minero a \\ la sostenibilidad en los países en \\ desarrollo (2018)\end{tabular}                                       & \begin{tabular}[c]{@{}l@{}}David\\ I. Stern\end{tabular}                                     & \begin{tabular}[c]{@{}l@{}}1980-2014\\ (Anuales)(19 países \\ en desarrollo no\\  pertenecientes a la\\  OPEP)\end{tabular} & VAR         & \begin{tabular}[c]{@{}l@{}}No es posible rechazar la hipótesis de \\ que el sector minero resta valor a la\\ sostenibilidad en los países en desarrollo \end{tabular}                                                 \\ \hline  \\[-1.8ex] 
\multicolumn{5}{l}{\footnotesize{\textit{Elaboración: Los autores}}} \\ 
\end{tabular}
\end{table}

\hypertarget{datos-y-metodologia}{%
\section{Datos y Metodología}\label{datos-y-metodologia}}

\hypertarget{datos}{%
\subsection{Datos}\label{datos}}

El crecimiento económico será representado mediante el producto bruto
interno real \emph{(PBI)}, dando la posibilidad de comparar la
producción real en tiempos diferentes, siendo una buena medida para el
crecimiento económico. El sector minero estará reflejado mediante los
datos anuales de valores FOB de la exportación de minerales metálicos
(cobre, zinc, oro, plata, estaño, hierro plomo y molibdeno), los cuales
se tomaron como una sola variable \emph{(XMIN)}, pues captura la
actividad minera en crecimiento y el sector industrial mediante el
Producto Bruto Interno Manufacturero \emph{(PBIMAN)}:

Los datos se obtuvieron de la base de datos del Banco Central de la
Republica del Perú, del periodo de 1994-2016 expresados en precios del
2011, ya que en esa época el sector minero fue un soporte esencial en la
económica peruana, pues fue un año de precios de los metales
excepcionalmente altos.

\hypertarget{metodologia}{%
\subsection{Metodología}\label{metodologia}}

La relación entre Exportaciones Mineras Metálicas, Producto Interno
Bruto Manufacturero y el Producto Interno Bruto se puede expresar de la
siguiente manera:

\begin{eqnarray}\label{eq1}
PBI_t=\propto_0+\alpha_1PBIMAN_t+\alpha_2XMIN_t+\varepsilon_T     
\end{eqnarray}

Donde:

\begin{itemize}
\tightlist
\item
  \({PBI}_t =\) Producto Interno Bruto
\item
  \({PBIMAN}_t =\) Producto Interno Bruto Manufacturero
\item
  \({XMIN}_t =\) Exportaciones Mineras Metálicas
\item
  \(\varepsilon_t =\) Ruido Blanco
\end{itemize}

Donde:

\begin{itemize}
\tightlist
\item
  \(t\): es la variable de tendencia.
\end{itemize}

Entonces, comenzamos el proceso de estimación a partir del modelo
\ref{eq1}. El propósito de este trabajo es examinar la relación a largo
plazo entre Exportaciones Mineras Metálicas, Producto Interno Bruto
Manufacturero y el Producto Interno Bruto en Perú. El enfoque
metodológico del estudio incluye los siguientes pasos:

\begin{enumerate}
\def\labelenumi{\arabic{enumi}.}
\tightlist
\item
  Comprobamos las propiedades de estacionariedad de la serie aplicando
  la prueba de Dickey-Fuller (ADF) aumentada (Dickey and Fuller 1981)
  así como la prueba de Phillips-Perron (PP)(Phillips and Perron 1988).
\item
  Si todas las variables están integradas de orden uno, entonces la
  prueba de cointegración de Johansen (1995) es la más apropiada para
  ser usada.
\item
  El tercer paso consiste en comprobar la relación causal entre las
  variables utilizando las pruebas apropiadas.
\end{enumerate}

En la Figura \ref{fig:plot1} se presenta la evolución de las tres
variables, Exportaciones Mineras Metálicas, Producto Interno Bruto
Manufacturero y el Producto Interno Bruto de Perú durante los años
1994-2016. Las estadísticas descriptivas de las variables se muestran en
la Cuadro \ref{tab:descript}.

\begin{figure}
\centering
\includegraphics{figs/graph1.pdf}
\caption{\label{fig:plot1} Representación gráfica de los datos de Perú
(Niveles)}
\end{figure}

\hypertarget{la-relacion-causal-entre-el-pib-exportaciones-mineras-metalicas-y-el-producto-interno-bruto-manufacturero-resultados-empiricos}{%
\section{La relación causal entre el PIB, Exportaciones Mineras
Metálicas y el Producto Interno Bruto Manufacturero: Resultados
Empíricos}\label{la-relacion-causal-entre-el-pib-exportaciones-mineras-metalicas-y-el-producto-interno-bruto-manufacturero-resultados-empiricos}}

\hypertarget{pruebas-de-raices-unitarias}{%
\subsection{Pruebas de Raíces
Unitarias}\label{pruebas-de-raices-unitarias}}

Aplicando las pruebas de raíz unitaria del ADF, de Dickey y Fuller
(Dickey and Fuller 1981), y del PP, de Phillips y Perron (Phillips and
Perron 1988), presentamos los resultados en el Cuadro \ref{tab:tb2}. Los
resultados del Cuadro \ref{tab:tb2} mostraron que todas las variables
son estacionarias en su primera diferencias (para la prueba de ADF), es
decir, están integradas de orden uno (es decir, \emph{I (\ref{eq1})}).
Por lo tanto, se continua aplicando el enfoque de cointegración de
Johansen para examinar la relación a largo plazo entre las variables.

\begin{table}[!htbp] \centering 
  \caption{Estadística Descriptiva para las Variables Usadas en el Análisis} 
  \label{tab:descript} 
\small 
\begin{tabular}{@{\extracolsep{5pt}}lccccc} 
\\[-1.8ex]\hline \\[-1.8ex] 
Statistic & \multicolumn{1}{c}{N} & \multicolumn{1}{c}{Mean} & \multicolumn{1}{c}{St. Dev.} & \multicolumn{1}{c}{Min} & \multicolumn{1}{c}{Max} \\ 
\hline 
\hline \\[-1.8ex] 
PBI & 23 & 309,856.20 & 105,249.60 & 182,043.70 & 502,341.30 \\ 
PBIMAN & 23 & 47,526.75 & 13,678.43 & 30,583.02 & 68,507.77 \\ 
XMIN & 23 & 11,991.95 & 9,187.39 & 1,959.70 & 27,494.47 \\ 
\hline 
\hline \\[-1.8ex] 
\multicolumn{6}{l}{\footnotesize{\textit{Nota:} Las estadísticas descriptivas corresponden}} \\ 
\multicolumn{6}{l}{\footnotesize{~~~~~~~~~~ al periodo 1990-2014}} \\ 
\multicolumn{6}{l}{\footnotesize{\textit{Elaboración: Los autores}}} \\ 
\end{tabular} 
\end{table}

\begin{table}[!htbp] \centering 
  \caption{Los resultados de la prueba de raíz de la unidad ADF sobre el crecimiento económico y el CO2 durante 1990-2014} 
  \label{tab:tb2} 
\begin{tabular}{@{\extracolsep{5pt}} lcclccl} 
\\[-1.8ex]\hline 
\hline \\[-1.8ex] 
\\[-1.8ex] &  \multicolumn{3}{c}{\textit{Dickey–Fuller}} &  \multicolumn{3}{c}{\textit{Phillips–Perron}}\\
\\[-1.8ex] Variables & t Valor & p & Estabilidad & t Valor & p & Estabilidad\\
\hline \\[-1.8ex] 
PBI & -1.168 & 0.889 & No Estacionario & -1.829 & 0.967 & No Estacionario \\ 
PBIMAN & -1.627 & 0.714 & No Estacionario & -6.334 & 0.721 & No Estacionario \\ 
XMIN & -1.498 & 0.764 & No Estacionario & -7.139 & 0.667 & No Estacionario \\ 
 &  &  &  &  &  &  \\ 
$\Delta$PBI & -4.157 & 0.018 & Estacionario & -20.681 & 0.02 & Estacionario \\ 
$\Delta$PBIMAN & -3.96 & 0.025 & Estacionario & -20.272 & 0.023 & Estacionario \\ 
$\Delta$XMIN & -3.853 & 0.032 & Estacionario & -12.212 & 0.327 & No Estacionario \\ 
\hline \\[-1.8ex] 
\multicolumn{7}{l}{\footnotesize{$\Delta$: Primera Diferencia}} \\ 
\multicolumn{7}{l}{\footnotesize{\textit{Elaboración: Los autores}}} \\ 
\end{tabular} 
\end{table}

\hypertarget{prueba-de-cointegracion}{%
\subsection{Prueba de Cointegración}\label{prueba-de-cointegracion}}

Dado que el enfoque de Johansen (1988) es sensible a la longitud del
retraso, antes de aplicar la prueba de cointegración, tenemos que
encontrar el orden del modelo VEC,la longitud de desfase óptima se
selecciona por el valor mínimo de los criterios AIC, SC, HQ y FPE. En la
Cuadro \ref{tab:tb3} se presentan los resultados de estos criterios.
Todos los criterios indican que la longitud óptima del rezago es igual a
2. Por lo tanto, el orden del modelo VEC es igual a 2.

La clave de la prueba de cointegración reside en la selección de la
forma adecuada de la prueba de cointegración y el orden de retraso. La
relación de cointegración entre las variables del modelo VEC se prueba
generalmente con el método de Johnsen (1988). Aquí las secuencias
seleccionadas son términos de tendencia lineal, y luego la forma de
prueba de la ecuación de cointegración es sólo de intercepción. El
número máximo de vectores de cointegración es el número de variables
menos uno (n-1=3-1=2).

La prueba de cointegración de Johansen sobre las Exportaciones Mineras,
PIB Manufacturero y el PIB, en el Cuadro \ref{tab:tb4} muestra que, en
la prueba de trazas los resultados de la prueba deben aceptar la
hipótesis nula, y existen dos relaciones positivas. Esto significa que
hay relaciones de equilibrio estables y a largo plazo entre las
variables. Partiendo de la premisa de la existencia de relaciones de
cointegración, la modelización del VEC puede realizarse más adelante.

\begin{table}[!htbp] \centering 
  \caption{Criterios de selección del orden de rezagos del VEC (Max=2)} 
  \label{tab:tb3} 
\small 
\begin{tabular}{@{\extracolsep{5pt}} lcc} 
\\[-1.8ex]\hline 
\hline \\[-1.8ex] 
Parámetros & Lag 1 & Lag 2 \\ 
\hline \\[-1.8ex] 
AIC(n) & 47.043 & 46.69$^{*}$ \\ 
HQ(n) & 47.172 & 46.917$^{*}$ \\ 
SC(n) & 47.64$^{*}$ & 47.735 \\ 
FPE(n) & 2.73e20 & 2.05e20$^{*}$ \\ 
\hline \\[-1.8ex] 
\multicolumn{3}{l}{\footnotesize{$^{*}$Indica el orden de retraso seleccionado   }} \\ 
\multicolumn{3}{l}{\footnotesize{por el criterio}} \\ 
\multicolumn{3}{l}{\footnotesize{\textit{Elaboración: Los autores}}} \\ 
\end{tabular} 
\end{table}

\begin{table}[!htbp] \centering 
  \caption{Los resultados de la prueba de cointegración del crecimiento económico, las exportaciones mineras y la producción industrial durante 1994-2016} 
  \label{tab:tb4} 
\begin{tabular}{@{\extracolsep{5pt}} clccc} 
\\[-1.8ex]\hline 
\hline \\[-1.8ex] 
 & Hipótesis & Trace Estadístico & p valor & Ecuac. de 
Cointegración \\ 
\hline \\[-1.8ex] 
1 & H0:r=0, H1:r\textgreater 0 & $53.701$ & \textless  0.001 & $0$ \\ 
2 & H0:r=1, H1:r \textgreater 1 & $23.972$ & 0.001637 & $0$ \\ 
3 & H0:r=2, H1:r \textgreater 2 & $4.178$ & 0.050960 & $2$ \\ 
\hline \\[-1.8ex] 
\multicolumn{5}{l}{\footnotesize{Las estadísticas indican 2 ecuaciones de cointegración al nivel del .01 de p-valor}} \\ 
\multicolumn{5}{l}{\footnotesize{\textit{Elaboración: Los autores}}} \\ 
\end{tabular} 
\end{table}

Los resultados del Cuadro \ref{tab:tb4} (estadísticas de pruebas de
trazas) apoyan la presencia de un vector cointegrador, concluimos que
hay una fuerte evidencia de cointegración a largo plazo entre las
variables examinadas. Las ecuaciones de cointegración a largo plazo se
muestran a continuación:

\begin{eqnarray}\label{eq2}
PBI_{t-1}=35.103XMIN_{t-1}+283577.505 \\
\label{eq3}
PBIMAN_{t-1}=2.308XMIN_{t-1}+33317.268
\end{eqnarray}

Basándose en la ecuación \ref{eq2} de cointegración anterior, el estudio
concluye que, los impactos a largo plazo de las Exportaciones Mineras en
el PIB son positivos y significativos, es decir, se mueven juntos en la
misma dirección. Mientras que para la ecuación \ref{eq3} de
cointegración se concluye que los impactos a largo plazo de las
Exportaciones Mineras en el PIB Manufacturero son positivos y
significativos, es decir, se mueven juntos en la misma dirección.

\hypertarget{estimacion-y-analisis-del-vecm}{%
\subsection{Estimación y análisis del
VECM}\label{estimacion-y-analisis-del-vecm}}

Después de la relación a largo plazo, continuamos aplicando el VECM para
determinar la dirección de causalidad entre las variables examinadas.
Las ecuaciones que se utilizan para probar la causalidad de Granger son
las siguientes:

\begin{equation}
\begin{bmatrix}
\Delta PIB_{t}\\ 
\Delta PIBMAN_{t}\\ 
\Delta XMIN_{t}
\end{bmatrix}=
\begin{bmatrix}
\alpha _{1}\\ 
\alpha _{2}\\ 
\alpha _{3}
\end{bmatrix}+
\sum_{i=1}^{p}
\begin{bmatrix}
\beta_{11}  & \beta_{21}  & \beta_{31}\\ 
\beta_{21} &  \beta_{21}  & \beta_{31}\\ 
\beta_{31} &  \beta_{21}  & \beta_{31}
\end{bmatrix}
\begin{bmatrix}
\Delta PIB_{t-i}\\ 
\Delta PIBMAN_{t-i}\\ 
\Delta XMIN_{t-i}
\end{bmatrix}+
\begin{bmatrix}
\lambda _{1}\\ 
\lambda _{2}\\ 
\lambda _{3}
\end{bmatrix}EMC_{t-1}+
\begin{bmatrix}
\mu _{1}\\ 
\mu _{2}\\ 
\mu _{3}
\end{bmatrix}
\label{eq:eq4}
\end{equation}

donde i(i=1, \ldots{} , p) es la longitud de retraso óptima determinada
por los criterios anteriormente analizados; \({ECM}_{t-1}\) es el
residuo rezagado obtenido de la relación a largo plazo presentada en la
ecuación \ref{eq1}; \(\lambda_1\),\(\lambda_2\) y \(\lambda_3\) son los
coeficientes de ajuste; y \(\mu_{1t}\), \(\mu_{2t}\) y \(\mu_{3t}\) son
los términos de perturbación que se supone que no están correlacionados
con cero, es decir, \(N(0, \sigma)\).

Los resultados del Cuadro \ref{tab:tb6} muestran que el grado de ajuste
del modelo VEC \(R^2\) \textgreater{} 0,5, y los valores de los
criterios AIC y SC son relativamente pequeños, lo que indica la
razonabilidad de la estimación del modelo y la línea de promedio cero
representa una relación de equilibrio estable y a largo plazo entre las
variables.

\begin{table}[!htbp] \centering 
  \caption{Los resultados de la estimación del modelo VEC} 
  \label{tab:tb6} 
\small 
\begin{tabular}{@{\extracolsep{5pt}} lccc} 
\\[-1.8ex]\hline 
\hline \\[-1.8ex] 
\\[-1.8ex] & $\Delta PBI$  &  $\Delta PBIMAN$ & $\Delta XMIN$\\
\hline \\[-1.8ex] 
ETC1 & -0.298 & -0.153$^{*}$ & -0.247$^{***}$ \\ 
ETC2 & 4.659 & 1.615$^{*}$ & 3.002$^{***}$ \\ 
Intercepto & -77623.26 & -17472.577 & -41863.975 \\ 
 $\Delta$PBI(-1) & 0.139 & 0.053 & -0.081 \\ 
$\Delta$PBIMAN(-1) & -5.308$^{**}$ & -1.937$^{***}$ & -2.103$^{***}$ \\ 
$\Delta$XMIN(-1) & 4.2$^{***}$ & 1.153$^{***}$ & 1.699$^{***}$ \\ 
$\Delta$PBI(-2) & 0.562 & 0.023 & 0.029 \\ 
$\Delta$PBIMAN(-2) & -4.067$^{**}$ & -1.188$^{*}$ & -1.513$^{***}$ \\ 
$\Delta$XMIN(-2) & 4.042$^{**}$ & 1.032$^{*}$ & 0.841$^{***}$ \\ 
\hline \\[-1.8ex] 
R$^{2}$ & 0.9257 & 0.7433 & 0.9142 \\ 
\hline \\[-1.8ex] 
Log likelihood &  & -513.9239 &  \\ 
AIC &  & 54.29239 &  \\ 
SC &  & 55.7362 &  \\ 
\hline \\[-1.8ex] 
\multicolumn{4}{l}{\footnotesize{\textit{Nota:} $^{***}$, $^{**}$ y $^{*}$ indica el nivel significativo al .01, .05 y .1}} \\ 
\multicolumn{4}{l}{\footnotesize{respectivamente}} \\ 
\multicolumn{4}{l}{\footnotesize{\textit{Elaboración: Los autores}}} \\ 
\end{tabular} 
\end{table}

Además, es útil investigar la presencia de una relación de causalidad de
Granger a corto y largo plazo entre las tres variables. La presencia de
cointegración nos permite utilizar la representación de corrección de
errores para investigar la relación de causalidad entre las variables.
El Cuadro \ref{tab:tb7} proporciona los resultados estimados de la
causalidad de Granger a través del modelo VEC que exhibe causalidad
tanto a corto como a largo plazo.

El Cuadro \ref{tab:tb7} muestra el término de corrección de errores para
la ecuación de cointegración con PIB, PIBMAN y XMIN como variables
dependientes. En el Cuadro \ref{tab:tb7}, se destaca que existe una
relación de causalidad en el sentido de Granger unidireccional entre las
Exportaciones Mineras Metálicas y el nivel de Producción Interna Bruta,
también hay una relación de causalidad en el sentido de Granger
unidireccional entre el PIB Manufacturero y el PIB, mientras que hay una
relación de causalidad en el sentido de Granger en una dirección de las
Exportaciones Mineras Metálicas hacia el PIB Manufacturero en las dos
direcciones.

Los recursos minerales son materias primas para el sector industrial de
la economía. Como resultado, actúa como una inversión de inventario en
el proceso de producción industrial. La expansión del crecimiento
económico requiere un mayor stock de capital y que a su vez requieren
una mayor tasa de inversión en inventario en el sector industrial de la
economía.

\begin{table}[!htbp] \centering 
  \caption{Los resultados de la causalidad de Granger} 
  \label{tab:tb7} 
\small 
\begin{tabular}{@{\extracolsep{5pt}} lcccl} 
\\[-1.8ex]\hline 
\hline \\[-1.8ex] 
\\[-1.8ex] VarDep &   &  &  &Causalidad\\
\\[-1.8ex] & $\Delta PBI$  &  $\Delta PBIMAN$ & $\Delta XMIN$ &\\
\hline \\[-1.8ex] 
$\Delta$PBI & - & 6.98945$^{**}$ & 13.51599$^{***}$ & $\Delta$PBI $\leftarrow$ $\Delta$PBIMAN, $\Delta$PBI $\leftarrow$XMIN \\ 
 &  & (0.0304) & (0.0012) &  \\ 
$\Delta$PBIMAN & 0.321026 & - & 10.062$^{***}$ & $\Delta$PBIMAN $\leftarrow$ $\Delta$XMIN \\ 
 & (0.8517) &  & (0.0065) &  \\ 
$\Delta$XMIN & 1.297252 & 39.44536$^{***}$ & - & $\Delta$XMIN $\leftarrow$ $\Delta$PBIMAN \\ 
 & (0.5228) & (0.0000) &  &  \\ 
\hline \\[-1.8ex] 
\multicolumn{5}{l}{\footnotesize{\textit{Nota: $^{***}$ y $^{**}$} indica el nivel significativo al .01 y .05, respectivamente; el t estadistico de cada coeficiente}} \\ 
\multicolumn{5}{l}{\footnotesize{ estimado está entre paréntesis; $\leftarrow$ denota una causalidad unidireccional; $\Delta$ denota primera diferencia.}} \\ 
\multicolumn{5}{l}{\footnotesize{\textit{Elaboración: Los autores}}} \\ 
\end{tabular} 
\end{table}

En nuestro análisis, hemos estimado el modelo, pero es necesario
realizar una verificación de robustez del modelo. La eficacia del modelo
ha sido probada para el supuesto de normalidad, heterocedasticidad y
presencia de correlación serial como se observa en el Cuadro
\ref{tab:tb5}. La distribución normal de los residuos se ha examinado en
función del resultado del estadístico de Jarque-Bera . El resultado del
diagnóstico sostiene que los residuos se distribuyen normalmente. Por lo
tanto, nos movemos para probar la presencia de heteroscedasticidad en
nuestro modelo basado en la prueba de heteroscedasticidad condicional
autorregresiva (ARCH). El resultado no revela ningún efecto ARCH
presente en el modelo que fortalezca los resultados estimados. Además,
la presencia de correlación en serie en el modelo se ha verificado a
través de la prueba de correlación en serie de Breusch-Godfrey, conocida
alternativamente como prueba de multiplicador de Lagrange (LM). Se ha
encontrado que no hay evidencia de correlación serial en el modelo. La
lectura de los resultados de las pruebas anteriores fortalece los
modelos estimados.

\begin{table}[!htbp] \centering 
  \caption{Resultados de las pruebas de diagnóstico del VEC(2)} 
  \label{tab:tb5} 
\small 
\begin{tabular}{@{\extracolsep{5pt}} clcc} 
\\[-1.8ex]\hline 
\hline \\[-1.8ex] 
 & Test de Diagnóstico & Estadística & p-valor \\ 
\hline \\[-1.8ex] 
1 & Test J-B & 7.8761 & 0.2473 \\ 
2 & LM Autocorrelación & 84.698 & 0.9939 \\ 
3 & Heterocedasticidad & 90 & 1 \\ 
4 & Estabilidad & Raíces fuera del círculo unitario & - \\ 
\hline \\[-1.8ex] 
\multicolumn{4}{l}{\footnotesize{\textit{Nota:} La Ho nula de J-B es la normalidad residual, la Ho es no autocorrelación, }} \\ 
\multicolumn{4}{l}{ \footnotesize{ la Ho de la prueba ARCH es heterocedasticidad, la estabilidad VEC revela que}} \\ 
\multicolumn{4}{l}{ \footnotesize{  5 raíces tienen un módulo menor que uno y se encuentran dentro del círculo de }} \\ 
\multicolumn{4}{l}{\footnotesize{la unidad. }} \\ 
\multicolumn{4}{l}{\footnotesize{\textit{Elaboración: Los autores}}} \\ 
\end{tabular} 
\end{table}

El modelo posee tres raíces fuera del circulo unitario, por
consiguiente, el modelo no es estable, lo cual afectaría la
confiabilidad de la capacidad predictiva del mismo, procedemos a
calcular dicha predicción (\ref{fig:plot2}), y existe
una divergencia entre los resultados del modelo y la evidencia empírica,
se destaca el hecho que según el modelo, las exportaciones de minerales
metálicos disminuirán en los años posteriores a los analizados, cuando
en la realidad, tuvieron un aumento significativo, según el Ministerio
de Energía y Minas del Perú y en el año 2017 representaron el 60\% de
todas las exportaciones, que sobrepasa el 55\% que tuvieron en el 2016
(MINEM 2018).

\begin{figure}[h]
\centering
\includegraphics[width=0.7\textwidth,height=\textheight]{C:/Users/crist/Documents/eco2/trabajo_eco2/doc/figs/pred.pdf}
\caption{\label{fig:plot2} Previsión del modelo (En t+1)}
\end{figure}

\hypertarget{conclusiones}{%
\section{Conclusiones y Recomendaciones}\label{conclusiones}}

Los resultados sugieren que las exportaciones mineras metálicas han sido
determinantes para el crecimiento económico del Perú en el periodo
1994-2016. A pesar de que las divisas generadas por la exportación de
minerales metálicos ayudan en gran medida lograr una mayor producción
manufacturera, la orientación de inversiones hacia el sector minero
pudieron haber ocasionado que el sector manufacturero no se expanda a un
ritmo que le permita liderar el crecimiento económico.

Además, se concluyó que existió una relación de causalidad de las
exportaciones mineras metálicas hacia el PBI, es decir, las
exportaciones metálicas impulsaron el crecimiento económico del Perú.
Esto respalda el equilibrio a largo plazo entre la minería metálica y el
crecimiento económico peruano (Ecuación \ref{eq2} y \ref{eq3}).
Asimismo, se evidenció una causalidad de la producción industrial hacia
el PIB (Cuadro \ref{tab:tb7}).

La estrategia actual con la exportación de minerales como medio de
ingresos de divisas no es viable ni justificable con respecto a lograr
un desarrollo sostenible para la republica peruana. Los recursos
minerales son de naturaleza agotable que no podrían reponerse dentro de
un horizonte temporal económicamente factible. Por lo tanto, la
extracción de minerales requiere un proceso de planificación sostenible
así que el Gobierno del Perú debería tomar las medidas necesarias para
extraer los recursos minerales sin comprometer el objetivo a largo plazo
de la sostenibilidad.


\hypertarget{referencias}{%
\section{Referencias}\label{referencias}}

\setlength{\parindent}{-0.2in}
\setlength{\leftskip}{0.2in}
\setlength{\parskip}{8pt}
\vspace*{-0.2in}

\noindent

\hypertarget{refs}{}
\leavevmode\hypertarget{ref-Dickey1981}{}%
Dickey, David A., and Wayne A. Fuller. 1981. ``Likelihood Ratio
Statistics for Autoregressive Time Series with a Unit Root.''
\emph{Econometrica} 49 (4): 1057. \url{https://doi.org/10.2307/1912517}.

\leavevmode\hypertarget{ref-Koitsiwe2015}{}%
Koitsiwe, Kegomoditswe, and Tsuyoshi Adachi. 2015. ``Relationship
between mining revenue, government consumption, exchange rate and
economic growth in Botswana.'' \emph{Contaduria Y Administracion} 60:
133--48. \url{https://doi.org/10.1016/j.cya.2015.08.002}.

\leavevmode\hypertarget{ref-MINEM2018}{}%
MINEM, MINISTERIO DE ENERGÍA y MINAS. 2018. ``CUARTO INFORME TRIMESTRAL
OCTUBRE-NOVIEMBRE-DICIEMBRE-2018.'' Lima: MINISTERIO DE ENERGÍA Y MINAS.

\leavevmode\hypertarget{ref-Phillips1988}{}%
Phillips, Peter C. B., and Pierre Perron. 1988. ``Testing for a Unit
Root in Time Series Regression.'' \emph{Biometrika} 75 (2): 335.
\url{https://doi.org/10.2307/2336182}.

\leavevmode\hypertarget{ref-Sahoo2014}{}%
Sahoo, Auro Kumar, Dukhabandhu Sahoo, and Naresh Chandra Sahu. 2014.
``Mining export, industrial production and economic growth: A
cointegration and causality analysis for India.'' \emph{Resources
Policy} 42: 27--34.
\url{https://doi.org/10.1016/j.resourpol.2014.09.001}.

\leavevmode\hypertarget{ref-Siliverstovs2007}{}%
Siliverstovs, Boriss, and Dierk Herzer. 2007. ``Manufacturing exports,
mining exports and growth: Cointegration and causality analysis for
Chile (1960-2001).'' \emph{Applied Economics} 39 (2): 153--67.
\url{https://doi.org/10.1080/00036840500427965}.

\leavevmode\hypertarget{ref-Stern2018}{}%
Stern, David I. 2018. ``` Mechanism Between Mining Sector and Economic
Growth in Zimbabwe , Is It a Resource Curse ?'.'' \emph{Environmental
Economics} 6 (3): 1--13.




\newpage
\singlespacing 
\end{document}
